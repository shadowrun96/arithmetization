% !TEX TS-program = pdflatex
% !TEX encoding = UTF-8 Unicode

% This is a simple template for a LaTeX document using the "article" class.
% See "book", "report", "letter" for other types of document.

\documentclass[11pt]{article} % use larger type; default would be 10pt

\usepackage[utf8]{inputenc} % set input encoding (not needed with XeLaTeX)

%%% Examples of Article customizations
% These packages are optional, depending whether you want the features they provide.
% See the LaTeX Companion or other references for full information.

%%% PAGE DIMENSIONS
\usepackage{geometry} % to change the page dimensions
\geometry{a4paper} % or letterpaper (US) or a5paper or....
% \geometry{margin=2in} % for example, change the margins to 2 inches all round
% \geometry{landscape} % set up the page for landscape
%   read geometry.pdf for detailed page layout information

\usepackage{graphicx} % support the \includegraphics command and options
\usepackage{amsmath}

% \usepackage[parfill]{parskip} % Activate to begin paragraphs with an empty line rather than an indent

%%% PACKAGES
\usepackage{booktabs} % for much better looking tables
\usepackage{array} % for better arrays (eg matrices) in maths
\usepackage{paralist} % very flexible & customisable lists (eg. enumerate/itemize, etc.)
\usepackage{verbatim} % adds environment for commenting out blocks of text & for better verbatim
\usepackage{subfig} % make it possible to include more than one captioned figure/table in a single float
% These packages are all incorporated in the memoir class to one degree or another...

%%% HEADERS & FOOTERS
\usepackage{fancyhdr} % This should be set AFTER setting up the page geometry
\pagestyle{fancy} % options: empty , plain , fancy
\renewcommand{\headrulewidth}{0pt} % customise the layout...
\lhead{}\chead{}\rhead{}
\lfoot{}\cfoot{\thepage}\rfoot{}

%%% SECTION TITLE APPEARANCE
\usepackage{sectsty}
\allsectionsfont{\sffamily\mdseries\upshape} % (See the fntguide.pdf for font help)
% (This matches ConTeXt defaults)

%%% ToC (table of contents) APPEARANCE
\usepackage[nottoc,notlof,notlot]{tocbibind} % Put the bibliography in the ToC
\usepackage[titles,subfigure]{tocloft} % Alter the style of the Table of Contents
\renewcommand{\cftsecfont}{\rmfamily\mdseries\upshape}
\renewcommand{\cftsecpagefont}{\rmfamily\mdseries\upshape} % No bold!

%%% END Article customizations

%%% The "real" document content comes below...

% Set Matrix Column to 16
\setcounter{MaxMatrixCols}{16}
\title{Arithmetization of Boolean Gates}
\author{George Fountis}
%\date{} % Activate to display a given date or no date (if empty),
         % otherwise the current date is printed 

\begin{document}
\maketitle

\section{Arithmetic}

\subsection{NOT}

$$ \textbf{0} \cdot  -1 + 1 = \textbf{1}$$
$$ \textbf{1} \cdot  -1 + 1 = \textbf{0}$$

\subsection{AND}

$$ 0 \cdot 0 = 0$$
$$ 0 \cdot 1 = 0$$
$$ 1 \cdot 0 = 0$$
$$ 1 \cdot 1 = 1$$

\subsection{NAND}
$$ (\textbf{0} \cdot \textbf{0}) \cdot -1+1 = \textbf{1}$$
$$ (\textbf{0} \cdot \textbf{1}) \cdot -1+1 = \textbf{1}$$
$$ (\textbf{1} \cdot \textbf{0}) \cdot -1+1 = \textbf{1}$$
$$ (\textbf{1} \cdot \textbf{1}) \cdot -1+1 = \textbf{0}$$

\subsection{OR}

$$ ((   \textbf{0}\cdot-1+1)  \cdot   (\textbf{0} \cdot -1+1))\cdot-1+1=\textbf{0}$$
$$ ((    \textbf{0}  \cdot-1+1)\cdot(  \textbf{1}  \cdot-1+1))\cdot-1+1=\textbf{1}$$
$$ ((    \textbf{1}  \cdot-1+1)\cdot(  \textbf{0}  \cdot-1+1))\cdot-1+1=\textbf{1}$$
$$ ((    \textbf{1}  \cdot-1+1)\cdot(   \textbf{1}   \cdot-1+1))\cdot-1+1=\textbf{1}$$


\newpage
\section{Multi-bit gates as vectors}
$$ a = (0,0,0,0,0,0,0,0)$$
$$ b = (1,1,1,1,1,1,1,1)$$

%$$ u = (0, 1, 0, 0, 0, 0, 0, 1)$$
$$ k = (-1, -1, -1, -1, -1, -1, -1, -1)$$
$$ v = (1, 1, 1, 1, 1, 1, 1, 1)$$
\subsection{NOT}

$$ (a \cdot k) + v$$

\subsection{AND}
$$ a\cdot b$$

\subsection{OR}
$$ (   (a \cdot k + v) \cdot (b \cdot k +v) ) \cdot k  + v$$
\marginpar{unchecked}



\section{Multi-bit gates as matrices}
\subsection{8-bit NOT Gate}
Invert the bitstring 01000001 which is the decimal value 65, corresponding to the ASCII value of 'A'.

\begin{equation}
\nonumber
  \begin{bmatrix}
   0 & 1 & 0 & 0 & 0 & 0 & 0 & 1   
  \end{bmatrix}
  %
\cdot
  \begin{bmatrix}
    -1 \\
    -1 \\
    -1 \\
    -1 \\
    -1 \\
    -1 \\
    -1 \\
    -1 \\
  \end{bmatrix} 
%
+
  \begin{bmatrix}
   1 & 1 & 1 & 1 & 1 & 1 & 1 & 1   
 \end{bmatrix} 
\end{equation}

\subsection{Scalar product representation}

\begin{equation}
A = 
\begin{bmatrix}
0 & 1 & 0 & 0 & 0 & 0 & 0 & 1   
  \end{bmatrix}
\end{equation}


\begin{equation}
P = 
\begin{bmatrix}
   1 & 1 & 1 & 1 & 1 & 1 & 1 & 1   
\end{bmatrix}
\end{equation}

\begin{equation}
k = -1
\end{equation}

\begin{equation}
kA +  P
\end{equation}

\end{document}
